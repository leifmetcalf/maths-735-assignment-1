\documentclass[a4paper, 12pt]{article}
\usepackage[utf8]{inputenc}
\usepackage{mathtools}
\usepackage{amssymb}
\usepackage{enumitem}
\usepackage{parskip}
\usepackage{xfrac}

\newcommand{\N}{\mathbb{N}}
\newcommand{\R}{\mathbb{R}}
\newcommand{\Z}{\mathbb{Z}}
\newcommand{\Q}{\mathbb{Q}}
\newcommand{\half}{\sfrac{1\!}2}
\DeclarePairedDelimiter\abs{\lvert}{\rvert}
\DeclareMathOperator{\GL}{GL}
\DeclareMathOperator{\interior}{int}
\DeclareMathOperator{\closure}{cl}
\DeclareMathOperator{\frontier}{fr}

\setlist[enumerate, 1]{leftmargin=0pt, label=\textbf{\arabic*.}}

\begin{document}

\begin{enumerate}

\item \begin{enumerate}

\item \(\otimes^2V^*\) equipped with pointwise addition forms a vector space. The identity element is \(0_{\otimes^2V^*}(x)\coloneqq0\). Let \(R,S,T\) be bilinear forms on \(V\) and let \(a,b\in\R\). The vector space axioms follow:
\begin{gather*}
R+(S+T)=(R+S)+T\\
R+S=S+R\\
R+0_{\otimes^2V^*}=R\\
R+-R=0_{\otimes^2V^*}\\
a(bR)=(ab)R\\
1R=R\\
a(R+S)=aR+aS\\
(a+b)R=aR+bR
\end{gather*}

\item The inverse of a symmetric bilinear form is a symmetric bilinear form, and the addition of two symmetric bilinear forms is a symmetric bilinear form.

\item Clearly \(\Lambda^2V^*\) and \(S^2V^*\) are disjoint. Let \(T\in\otimes^2V^*\). Define
\begin{gather*}
R(v,u)\coloneqq\half\big(T(u,v)-T(v,u)\big)\\
S(u,v)\coloneqq\half\big(T(u,v)+T(v,u)\big)
\end{gather*}
Then \(T=R+S\) and \(R\in\Lambda^2V^*\) and \(S\in S^2V^*\), so \(\otimes^2V^*=\Lambda^2V^*+S^2V^*\) and hence \(\otimes^2V^*=\Lambda^2V^*\oplus S^2V^*\).

\item 

\end{enumerate}

\end{enumerate}

\end{document}