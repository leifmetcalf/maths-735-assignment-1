\documentclass[a4paper, 12pt]{article}
\usepackage[utf8]{inputenc}
\usepackage{mathtools}
\usepackage{amssymb}
\usepackage{enumitem}
\usepackage{parskip}
\usepackage{xfrac}

\newcommand{\tr}{^{\mathsf{T}}}
\newcommand{\N}{\mathbb{N}}
\newcommand{\R}{\mathbb{R}}
\newcommand{\Z}{\mathbb{Z}}
\newcommand{\Q}{\mathbb{Q}}
\newcommand{\half}{\sfrac{1\!}2}
\DeclarePairedDelimiter\abs{\lvert}{\rvert}
\DeclareMathOperator{\GL}{GL}
\DeclareMathOperator{\interior}{int}
\DeclareMathOperator{\closure}{cl}
\DeclareMathOperator{\sgn}{sgn}

\DeclareFontFamily{U}{mathb}{\hyphenchar\font45}
\DeclareFontShape{U}{mathb}{m}{n}{
      <5> <6> <7> <8> <9> <10> gen * mathb
      <10.95> mathb10 <12> <14.4> <17.28> <20.74> <24.88> mathb12
      }{}
\DeclareSymbolFont{mathb}{U}{mathb}{m}{n}
\DeclareFontSubstitution{U}{mathb}{m}{n}
\let\dddot\relax
\DeclareMathAccent{\dddot}{0}{mathb}{"3B}

\setlist[enumerate, 1]{leftmargin=0pt, label=\textbf{\arabic*.}}

\begin{document}

\begin{enumerate}

\item \begin{enumerate}

\item \(\otimes^2V^*\) is a subspace of the vector space of functions \(V\times V\to\R\) since the identity element \(0_{\otimes^2V^*}(x)\coloneqq0\) is a bilinear form, the sum of two bilinear forms is a bilinear form, and the negation of a bilinear form is a bilinear form.

\item \(0_{\otimes^2V^*}\) is symmetric, the negation of a symmetric bilinear form is symmetric, and the sum of two symmetric bilinear forms is symmetric; so symmetric bilinear forms are a subspace of \(\otimes^2V^*\).

\item Clearly \(\Lambda^2V^*\cap S^2V^*=\{0\}\). Let \(T\in\otimes^2V^*\). Define
\begin{gather*}
K(v,u)\coloneqq\half\big(T(u,v)-T(v,u)\big)\\
H(u,v)\coloneqq\half\big(T(u,v)+T(v,u)\big)
\end{gather*}
Then \(T=K+H\) and \(K\in\Lambda^2V^*\) and \(H\in S^2V^*\), so \(\otimes^2V^*=\Lambda^2V^*+S^2V^*\) and hence \(\otimes^2V^*=\Lambda^2V^*\oplus S^2V^*\).

\item Let \(\{e_1,e_2\}\) be a basis for \(V\). Define the bilinear form
\[K(ae_1+be_2,ce_1+de_2)\coloneqq ad-bc.\]
Let \(T\in\Lambda^2V^*\). Let \(ae_1+be_2,ce_1+de_2\in V\). Then by bilinearity
\begin{align*}
&T(ae_1+be_2,ce_1+de_2)\\
&\quad=ac\cdot T(e_1,e_1)+ad\cdot T(e_1,e_2)+bc\cdot T(e_2,e_1)+bd\cdot T(e_2,e_2).
\end{align*}
But \(T\) is skew so \(T(e_1,e_1)=T(e_2,e_2)=0\) and \(T(e_1,e_2)=-T(e_2,e_1)\), and \(T\) simplifies to
\[T(ae_1+be_2,ce_1+de_2)=(ad-bc)T(e_1,e_2).\]
Hence \(T\) is a constant multiple of \(K\) and so \(\dim(\Lambda^2V^*)=1\).
\item

\end{enumerate}

\item \begin{enumerate}

\item We have
\[\dot c(t)=(2t,3t^2)\]
and hence on \(\R\setminus\{0\}\) we have
\begin{align*}
e_1(t)&=\frac1{\sqrt{4t^2+9t^4}}\begin{pmatrix}2t\\3t^2\end{pmatrix}\\
e_2(t)&=\frac1{\sqrt{4t^2+9t^4}}\begin{pmatrix}-3t^2\\2t\end{pmatrix}.
\end{align*}
Note
\[e_1(t)=\frac1{\sqrt{4+9t^2}}\begin{pmatrix}2\sgn t\\3\abs{t}\end{pmatrix}.\]
But there's no possible value that makes \(\sgn t\) continuous on \(\R\), so there cannot be a distinguished Frenet frame for \(c\) on all of \(\R\).

\item Note
\begin{align*}
\dot c(t)&=\abs{\dot c(t)}e_1(t)\\
\ddot c(t)&=D\big[\abs{\dot c(t)}\big]e_1(t)+\abs{\dot c(t)}\dot e_1(t)\\
&=D\big[\abs{\dot c(t)}\big]e_1(t)+\abs{\dot c(t)}^2\kappa(t)e_2(t)
\end{align*}
so
\[\det(\dot c(t),\ddot c(t))=\abs{\dot c(t)}^3\kappa(t)\]
and hence
\[\kappa(t)=\frac{\det(\dot c(t),\ddot c(t))}{\abs{\dot c(t)}^3}.\]

\item

\end{enumerate}

\item As above we have
\begin{align*}
\dot c(t)&=\abs{\dot c(t)}e_1(t)\\
\ddot c(t)&=\abs{\dot c(t)}\dot e_1(t)+o_1=\abs{\dot c(t)}^2\kappa(t)e_2(t)+o_1\\
\dddot c(t)&=\abs{\dot c(t)}^2\kappa(t)\dot e_2(t)+o_2=\abs{\dot c(t)}^3\kappa(t)\tau(t)e_3(t)+o_2
\end{align*}
where \(o_i\) is some vector in the span of \(e_1(t),\dots,e_i(t)\) which we ignore because it doesn't affect the determinants
\begin{align*}
\det(\dot c(t),\ddot c(t))&=\abs{\dot c(t)}^3\kappa(t)\\
\det(\dot c(t),\ddot c(t),\dddot c(t))&=\abs{\dot c(t)}^6\kappa(t)^2\tau(t).
\end{align*}
Hence
\begin{gather*}
\kappa(t)=\frac{\abs{\dot c(t)\times\ddot c(t)}}{\abs{\dot c(t)}^3}\\
\tau(t)=\frac{\det(\dot c(t),\ddot c(t),\dddot c(t))}{\abs{\dot c(t)\times\ddot c(t)}^2}.
\end{gather*}

\item \begin{enumerate}

\item

\item

\end{enumerate}

\item I don't know how to prove \(\dot R(t)\) is an endomorphism. Let \(t\in\R\) and \(u,v\in\R^n\). Since \(R(t)\) is a orthogonal transformation we have
\[R(t)u\cdot R(t)v=u\cdot v.\]
Differentiating both sides gives
\[\dot R(t)u\cdot R(t)v+R(t)u\cdot\dot R(t)v=0\]
and so since \(R(0)=\text{id}_{\R^n}\) we have
\[\dot R(t)u\cdot v+u\cdot\dot R(t)v=0\]

\item Note
\[2\langle u,v\rangle=\abs{u+v}^2-\abs{u}^2+\abs{v}^2\]

\end{enumerate}

\end{document}